\documentclass[letterpaper,12pt]{ctexart}
%-------------------------
% Resume in Latex
% Author : Aras Gungore
% License : MIT
%------------------------
% \documentclass[letterpaper,11pt]{ctexart}
\usepackage{latexsym}
\usepackage[empty]{fullpage}
\usepackage{titlesec}
\usepackage{marvosym}
\usepackage[usenames,dvipsnames]{color}
\usepackage{verbatim}
\usepackage{enumitem}
\usepackage{xcolor}
\definecolor{mydarkblue}{rgb}{0,0.08,0.45}
\usepackage[colorlinks=true, urlcolor=mydarkblue]{hyperref}
\usepackage{fancyhdr}
\usepackage[english]{babel}
\usepackage{tabularx}
\usepackage{hyphenat}
\usepackage{fontawesome}
\usepackage{setspace}
\usepackage{url}
\usepackage{array}
% \input{glyphtounicode}

\usepackage[numbers]{natbib}
\usepackage{bibentry}
\bibliographystyle{abbrv}

% \setlength{\footskip}{4.08003pt}

%---------- FONT OPTIONS ----------
% sans-serif
% \usepackage[sfdefault]{FiraSans}
% \usepackage[sfdefault]{roboto}
% \usepackage[sfdefault]{noto-sans}
% \usepackage[default]{sourcesanspro}

% serif
% \usepackage{CormorantGaramond}
% \usepackage{charter}

\usepackage[margin=0.6in]{geometry}
\pagestyle{fancy}
\fancyhf{} % clear all header and footer fields
\fancyfoot[C]{\makebox[\textwidth][c]{\thepage}} % explicitly center the page number in the footer
\renewcommand{\headrulewidth}{0pt}
\renewcommand{\footrulewidth}{0pt}
\setlength{\footskip}{20pt} % Increase or decrease the value to adjust the distance

% Adjust margins
% \addtolength{\oddsidemargin}{-0.5in}
% \addtolength{\evensidemargin}{-0.5in}
% \addtolength{\textwidth}{1in}
% \addtolength{\topmargin}{-.5in}
% \addtolength{\textheight}{1.0in}

\urlstyle{same}

\raggedbottom
\raggedright
\setlength{\tabcolsep}{0in}

\setlength{\parskip}{0pt}          % Remove the space between paragraphs
\setlength{\lineskip}{0pt}         % Remove the minimum space between lines
\setlength{\lineskiplimit}{0pt}    % Remove the limit for the minimum space between lines
\setstretch{1.1}                     % Set the line spacing to single spacing


% Sections formatting
\titleformat{\section}{
  \vspace{-4pt}\scshape\raggedright\normalsize\bfseries
}{}{0em}{}[\color{black}\titlerule \vspace{-5pt}]

% Ensure that generate pdf is machine readable/ATS parsable
% \pdfgentounicode=1

%-------------------------
% Custom commands

\newcommand{\resumeItem}[1]{
  \item\small{
    {#1 \vspace{-2pt}}
  }
}

\newcommand{\paperItem}[2]{
  \item[#1]\small{
    {#2 \vspace{-5pt}}
  }
}

\newcommand{\resumeSubheading}[4]{
  \vspace{-4pt}\item
    \begin{tabular*}{0.97\textwidth}[t]{l@{\extracolsep{\fill}}r}
      \textbf{\small#1} & \small#2 \\
      {\small#3} & {\small #4} \\
    \end{tabular*}\vspace{-7pt}
}

\newcommand{\resumeSubheadingDBRole}[6]{
  \vspace{-4pt}\item
    \begin{tabular*}{0.97\textwidth}[t]{l@{\extracolsep{\fill}}r}
      \textbf{\small#1} & \small#2 \\
      {\small#3} & {\small #4} \\
      {\small#5} & {\small #6} \\
    \end{tabular*}\vspace{-7pt}
}


\newcommand{\resumeSubSubheading}[2]{
    \vspace{-2pt}\item
    \begin{tabular*}{0.97\textwidth}{l@{\extracolsep{\fill}}r}
      {\small#1} & {\small #2} \\
    \end{tabular*}\vspace{-7pt}
}


\newcommand{\resumeEducationHeading}[6]{
  \vspace{-2pt}\item
    \begin{tabular*}{0.97\textwidth}[t]{l@{\extracolsep{\fill}}r}
      \textbf{\small#1} & \small #2 \\
      {\small#3} & {\small #4} \\
      {\small#5} & {\small #6} \\
    \end{tabular*}\vspace{-5pt}
}

\newcolumntype{P}[1]{>{\raggedleft\arraybackslash}p{#1}}

\newcommand{\resumeMentorHeading}[3]{
    \item
    \begin{tabular*}{0.97\textwidth}[t]{p{0.2\textwidth}p{0.57\textwidth}P{0.2\textwidth}}
      \small #1 & \small #2 &{\small #3}\\
    \end{tabular*}\vspace{-8pt}
}

\newcommand{\resumeAwardHeading}[3]{
    \item
    \begin{tabular*}{0.97\textwidth}[t]{p{0.46\textwidth}p{0.39\textwidth}P{0.12\textwidth}}
      \small\textbf{#1} & \small #2 &{\small #3}\\
    \end{tabular*}\vspace{-8pt}
}

\newcommand{\resumeProjectHeading}[2]{
    \vspace{-7pt}\item
    \begin{tabular*}{0.97\textwidth}{l@{\extracolsep{\fill}}r}
      \hspace{13pt}\small#1 & #2 \\
    \end{tabular*}\vspace{-7pt}
}


\newcommand{\resumeOrganizationHeading}[4]{
  \vspace{-2pt}\item
    \begin{tabular*}{0.97\textwidth}[t]{l@{\extracolsep{\fill}}r}
      \textbf{#1} & {\small #2} \\
      {\small#3}
    \end{tabular*}\vspace{-7pt}
}

\newcommand{\resumeSubItem}[1]{\resumeItem{#1}\vspace{-4pt}}

\renewcommand\labelitemii{$\vcenter{\hbox{\tiny$\bullet$}}$}

\newcommand{\resumeSubHeadingListStart}{\begin{itemize}[leftmargin=0.15in, label={}]}
\newcommand{\resumeSubHeadingListEnd}{\end{itemize}}
\newcommand{\resumeItemListStart}{\begin{itemize}}
\newcommand{\resumeItemListEnd}{\end{itemize}\vspace{-5pt}}

%-------------------------------------------
%%%%%%  RESUME STARTS HERE  %%%%%%%%%%%%%%%%%%%%%%%%%%%%


\begin{document}

%---------- HEADING ----------

\begin{center}
    \textbf{\large \scshape 个人简历} \\ \vspace{3pt}
    \small
    % \faMobile \hspace{.5pt} \href{tel:16176022921}{+1 (617)-602-2921}
    % $|$
    \faGlobe \hspace{.5pt} \href{https://dianwang.io}{dianwang.io}
    $|$
    \faTwitter \hspace{.5pt} \href{https://twitter.com/Dian_Wang_}{@Dian\_Wang\_}
    $|$
    \faGraduationCap \hspace{.5pt} \href{https://scholar.google.com/citations?user=CckjtfQAAAAJ&hl=en&authuser=1}{Google Scholar}
    \\
    \faYoutubePlay  \hspace{.5pt} \href{https://www.youtube.com/channel/UCsXEbqvtDnbtl_W01SYSeUQ}{Youtube}
    $|$
    \faGithub \hspace{.5pt} \href{https://github.com/pointW}{GitHub}
    $|$
    \faAt \hspace{.5pt} \href{mailto:wang.dian@northeastern.edu}{wang.dian@northeastern.edu}
    % \faLinkedinSquare \hspace{.5pt} \href{https://www.linkedin.com/in/arasgungore}{LinkedIn}
    % $|$
    % $|$
    % $|$
    % \faMapMarker \hspace{.5pt} \href{https://www.google.com/maps/place/Bogazici+University+North+Campus/@41.0863067,29.0441352,15z/data=!4m5!3m4!1s0x0:0x9d2497b07c8edb2f!8m2!3d41.0863067!4d29.0441352}{Istanbul, Turkey}
\end{center}

\section{个人信息}
\resumeSubHeadingListStart
\item
\small
\begin{tabular*}{0.97\textwidth}[t]{p{0.97\textwidth}}
姓名: 王点 \hfill 性别: 男 \hfill 出生年月: 1994.10\\
研究方向:具身智能、人工智能、机器人学习、几何深度学习、机器人操纵与抓取、强化学习
\end{tabular*}
\resumeSubHeadingListEnd

% \section{研究方向}
% \resumeSubHeadingListStart
% \item
% \small
% \begin{tabular*}{0.97\textwidth}[t]{p{0.97\textwidth}}
% 具身智能、机器人学习、几何深度学习、机器人操纵与抓取、强化学习 \\
% \end{tabular*}
% \resumeSubHeadingListEnd

%----------- EDUCATION -----------

\section{教育背景}
  % \vspace{3pt}
  \resumeSubHeadingListStart
    
    \resumeEducationHeading
      {美国东北大学
      % \normalfont{(Admission rate: 0.09\%)}
      }{美国马萨诸塞州波士顿}
      {计算机科学博士。指导老师: Prof. Robert Platt, Prof. Robin Walters}{2020.01 \textbf{--} 至今}
      {计算机科学硕士。   {GPA: 4.00/4.00}}{2017.09 \textbf{--} 2019.12}
        % \resumeItemListStart
            % \resumeItem{\textbf{Relevant coursework:} Calculus I-II, Matrix Theory, Differential Equations, Materials Science, Electrical Circuits I-II, Digital System Design, Numerical Methods, Probability Theory, Electronics I-II, Signals and Systems, Electromagnetic Field Theory, Energy Conversion, System Dynamics and Control, Communication Engineering}
        % \resumeItemListEnd
    
    \resumeSubheading
      {四川大学
      % \normalfont{(Admission rate: 0.85\%)}
      }{中国四川省成都市}
      {计算机科学与技术学士。   {GPA: 3.56/4.00}}{2013.09 \textbf{--} 2017.06}
\vspace{1pt}
  \resumeSubHeadingListEnd



%----------- RESEARCH EXPERIENCE -----------
\section{工作经历}
\resumeSubHeadingListStart
\vspace{2pt}

% \resumeSubheading
% {Northeastern University}{Boston, MA, USA}
% {Research Assistant}{Jan. 2018 \textbf{--} Present}
% \vspace{0.5pt}

% \resumeItemListStart
% \resumeItem{Proposed symmetric neural network architectures for improving training efficiency in robotic manipulation tasks.}
% \resumeItem{Implemented an open-sourced robotic reinforcement learning environment library using PyBullet.}
% \resumeItem{Built an assistive robotic system to assist people with disabilities in household manipulation tasks.}
% \resumeItemListEnd

% \textbf{Equivariant reinforcement learning in robotic manipulation}
% \resumeProjectHeading{\textbf{Equivariant learning in robotic manipulation}}{}
% \resumeItemListStart
% \resumeItem{Defined the symmetric properties of policy learning in robotic manipulation. }
% \resumeItem{Proposed symmetric neural network architectures for improving training efficiency in robotic manipulation tasks.}
% \resumeItemListEnd

% \resumeProjectHeading{\textbf{BulletArm reinforcement learning environments}}{}
% \resumeItemListStart
% \resumeItem{Implemented an open-sourced robotic reinforcement learning environment library using PyBullet.}
% \resumeItem{Built a real-world experimental platform using a UR5 arm.}
% \resumeItemListEnd

% % \resumeProjectHeading{\textbf{Policy learning in SE(3) action spaces}}{}
% % \resumeItemListStart
% % \resumeItem{Designed a reinforcement learning framework for robotic manipulation tasks.}
% % \resumeItem{Proposed an imitation learning algorithm for large action spaces.}
% % \resumeItemListEnd

% \resumeProjectHeading{\textbf{Assistive robotic pick-and-place system}}{}
% \resumeItemListStart
% \resumeItem{Built an assistive robotic system to assist people with disabilities in household manipulation tasks.}
% \resumeItem{Conducted pick-and-place experiments in an open world environment.}
% \resumeItemListEnd

\resumeSubheading
{波士顿动力人工智能研究院}{美国马萨诸塞州剑桥}
{科研实习生(科研岗)}{2024.05 \textbf{--} 2024.08}
\resumeSubheading
{波士顿动力人工智能研究院}{美国马萨诸塞州剑桥}
{科研实习生(科研岗)}{2023.05 \textbf{--} 2023.08}
% \resumeEducationHeading
% {Boston Dynamics AI Institute}{Cambridge, MA, USA}
% {Research Intern}{May 2023 \textbf{--} Aug. 2023}{Research Intern}{May 2024 \textbf{--} Aug. 2024}
% \resumeItemListStart
% \resumeItem{Proposed algorithms for solving long-horizon robotic manipulation tasks using geometric deep learning.}
% \resumeItemListEnd
% \vspace{0.5pt}

% \resumeSubheading
% {Institute of Computing Technology, Chinese Academy of Sciences}{Beijing, China}
% {Research Intern}{July 2016 \textbf{--} Aug. 2016}
% \textbf{Equivariant reinforcement learning in robotic manipulation}
% \resumeItemListStart
% \resumeItem{Led team of 4 interns to implement a user dynamic detection app based on data from gravity sensor.}
% \resumeItemListEnd
\vspace{1pt}
\resumeSubHeadingListEnd


\section{发表论文}
\nobibliography{dian_wang_cv_chinese}
\textsc{会议论文}
\vspace{-0.1cm}
\begin{enumerate}
\paperItem{C19}{\bibentry{wang2024diffusion}}
\paperItem{C18}{\bibentry{hu2024orbit}}
\paperItem{C17}{\bibentry{huang2024imagination}}
\paperItem{C16}{\bibentry{huang2023fourier}}
\paperItem{C15}{\bibentry{wang2023theory}}
\vspace{-0.5cm}
\paperItem{C14}{\bibentry{nguyen2023equivariant}}
\paperItem{C13}{\bibentry{wang2023surprising}}
\paperItem{C12}{\bibentry{jia2023seil}}
\paperItem{C11}{\bibentry{huang2023edge}}
\paperItem{C10}{\bibentry{wang2022onrobot}}
\paperItem{C9}{\bibentry{nguyen2022leveraging}}
\paperItem{C8}{\bibentry{wang2022bulletarm}}
\paperItem{C7}{\bibentry{huang2022equivariant}}
\paperItem{C6}{\bibentry{zhu2022sample}}
\paperItem{C5}{\bibentry{wang2022so2}}
\paperItem{C4}{\bibentry{wang2021equivariant}}
\paperItem{C3}{\bibentry{biza2021action}}
\paperItem{C2}{\bibentry{wang2020policy}}
\paperItem{C1}{\bibentry{wang2019assistive}}
\end{enumerate}
% \vspace{-0.3cm}
\textsc{期刊论文}
\vspace{-0.1cm}
\begin{enumerate}
\paperItem{J3}{\bibentry{huang2023leveraging}}
\paperItem{J2}{\bibentry{zhu2023grasp}}
\paperItem{J1}{\bibentry{wilkinson2021design}}
\end{enumerate}

\textsc{研讨会论文}
\vspace{-0.1cm}
\begin{enumerate}
% wang2024correctws
\paperItem{W9}{\bibentry{wang2024correctws}}
\paperItem{W8}{\bibentry{jia2024equivariantws}}
\paperItem{W7}{\bibentry{wang2023thews}}
\paperItem{W6}{\bibentry{jia2023seilws}}
\paperItem{W5}{\bibentry{huang2022edgews}}
\paperItem{W4}{\bibentry{wang2021equivariantws}}
\paperItem{W3}{\bibentry{wang2022equiws}}
\paperItem{W2}{\bibentry{zhu2022samplews}}
\paperItem{W1}{\bibentry{huang2022equivariantws}}
% \paperItem{P1}{\bibentry{klee2024reducing}}
\end{enumerate}

\textsc{预印论文}
\vspace{-0.1cm}
\begin{enumerate}
\paperItem{P3}{\bibentry{huang2024match}}
\paperItem{P2}{\bibentry{tangri2024equivariant}}
% \paperItem{P3}{\bibentry{zhu2024se3}}
\paperItem{P1}{\bibentry{jia2024open}}
% \paperItem{P1}{\bibentry{klee2024reducing}}
\end{enumerate}

% \vspace{3pt}
% \nocite{*} % Include all references from the .bib file
% \printbibliography

\section{荣誉奖项}
\vspace{-2pt}
\resumeSubHeadingListStart
\item
\small
\begin{tabular*}{0.97\textwidth}[t]{p{0.46\textwidth}p{0.39\textwidth}P{0.12\textwidth}}
\textbf{最佳论文提名奖} & 机器人学习会议 (CoRL) 2024 &{2024.11}\\
\textbf{2023 摩根大通博士奖学金} & 摩根大通 &{2023.06}\\
\textbf{最佳论文提名奖} & ICRA 2022 《机器人学习扩展》研讨会&{2022.05}\\
\textbf{东北大学Khoury科研奖学金} & 东北大学 &{2019.08}\\
% \textbf{First Place of Outstanding Bachelor’s Thesis} & Sichuan University &{June 2017}\\
\end{tabular*}
% \resumeAwardHeading{Best Paper Award Finalist}{ICRA 2022 Workshop on Scaling Robot Learning}{May 2022}
% \resumeAwardHeading{Khoury College Graduate Research Fellowship}{Northeastern University}{Aug. 2019}
% \resumeAwardHeading{First Place of Outstanding Bachelor’s Thesis}{Sichuan University}{June 2017}
\resumeSubHeadingListEnd

\section{学术服务}
\resumeSubHeadingListStart
\small{
\item \textbf{主要组织者}, RSS 2023《机器人学习中的对称性》研讨会
\vspace{-7pt}
\item \textbf{组织者}, RSS 2024《机器人学习中的几何与代数结构》研讨会
\vspace{-7pt}
\item \textbf{审稿人}: IJRR2024. ICML 2024. ICLR 2023-2025. NeurIPS 2023. ICRA 2019, 2022-2024. CoRL 2022-2024. IROS 2021, 2023. RAL 2022-2024. T-RO 2022.
}
% \vspace{-5pt}
\resumeSubHeadingListEnd


\section{教学经历}
\resumeSubHeadingListStart
\resumeSubheading{助教}{}{强化学习与序列决策(东北大学 CS5180),Chris Amato教授}{2024年秋季}
\clearpage
\resumeSubheading{特邀讲座:《机器人操作的等变策略学习》}{}{算法机器人学(莱斯大学 Comp550),Lydia Kavraki教授}{2024.11}
\resumeSubheading{特邀讲座:《机器人操作的等变强化学习》}{}{强化学习与序列决策(东北大学 CS5180),Lawson Wong教授}{2024.04}
\resumeSubheading{特邀讲座:《机器人操作的等变学习》}{}{几何深度学习(东北大学 CS7180),Robin Walters教授}{2023.04}
\resumeSubheading{特邀讲座:《机器人学习中利用SE(2)对称性》}{}{机器人科学与系统(东北大学 CS5335),Robert Platt教授}{2022.03}
\vspace{3pt}
\resumeSubHeadingListEnd

\section{学生指导}
\vspace{-2pt}
\resumeSubHeadingListStart
\item
\small
\begin{tabular*}{0.97\textwidth}[t]{p{0.13\textwidth}p{0.26\textwidth}p{0.38\textwidth}P{0.2\textwidth}}
Haibo Zhao & 东北大学硕士生 & 在读 &{2023.11 - 至今}\\
Mingxi Jia & 东北大学硕士生 & 现为布朗大学博士生 &{2021.12 - 2023.05}\\
Guanang Su & 东北大学硕士生 & 现为明尼苏达大学博士生 &{2021.12 - 2023.05}\\
Neel Sortur & 东北大学本科生 & 现为东北大学硕士生 & {2021.05 - 2022.10}\\
Zhengyi Ou & 东北大学硕士生 & 现为美敦力公司软件工程师 & {2020.09 - 2021.12}\\
Yida Niu & 东北大学硕士生 & 现为北京大学博士生 & {2020.09 - 2021.08}\\
\end{tabular*}
\resumeSubHeadingListEnd

%----------- AWARDS & ACHIEVEMENTS -----------

% \section{Media Coverage}
% \resumeSubHeadingListStart
% \item
% \small
% \begin{tabular*}{0.97\textwidth}[t]{p{0.85\textwidth}P{0.12\textwidth}}
% Khoury Story: Dian on Researching Machine Learning and Robotics, \href{https://www.youtube.com/watch?v=B9g2yhHs5Wg}{\underline{Link}} &{June 2024}\\
% Institute for Experiential Robotics Newsletter, Dian Wang - CoRL 2022 Presentation &{Jan. 2023}\\
% Northeastern Global News, photo by Matthew Modoono, \href{https://news.northeastern.edu/2020/09/28/machine-learning/}{\underline{Link}} &{Sept. 2020}\\
% IEEE Spectrum Video Friday, \href{https://spectrum.ieee.org/video-friday-misty-robotics-shipping-programmable-personal-robot}{\underline{Link}} &{Sept. 2019}\\
% \end{tabular*}
% \resumeSubHeadingListEnd

\section{媒体报道}
\resumeSubHeadingListStart
\item
\small
\begin{tabular*}{0.97\textwidth}[t]{p{0.85\textwidth}P{0.12\textwidth}}
Khoury Story采访:《王点的机器学习与机器人研究》,\href{https://www.youtube.com/watch?v=B9g2yhHs5Wg}{\underline{链接}} &{2024.06}\\
东北大学实验机器人研究所简报,《王点 - CoRL 2022演讲》 &{2023.01}\\
东北大学全球新闻,《每日新闻:机器学习》摄影:Matthew Modoono,\href{https://news.northeastern.edu/2020/09/28/machine-learning/}{\underline{链接}} &{2020.09}\\
IEEE Spectrum每周视频,《面向开放世界环境的辅助机器人抓取与放置》,\href{https://spectrum.ieee.org/video-friday-misty-robotics-shipping-programmable-personal-robot}{\underline{链接}} &{2019.09}\\
\end{tabular*}
\resumeSubHeadingListEnd

\section{公共服务}
\resumeSubHeadingListStart
\item
\small
\begin{tabular*}{0.97\textwidth}[t]{p{0.85\textwidth}P{0.12\textwidth}}
人工智能实践 - 日常机器人,在东北大学进行演示和展示 & {2024.04}\\
\end{tabular*}
\resumeSubHeadingListEnd

\section{学术报告与演讲}
\resumeSubHeadingListStart
\vspace{-4pt}\item
\begin{tabular*}{0.97\textwidth}[t]{l@{\extracolsep{\fill}}r}
\textbf{\small 《机器人操作的等变策略学习》} & \small  \\
% \small \textbf{(Equivariant Policy Learning for Robotic Manipulation)} & \\
{\small 清华大学} & {\small 2024.12} \\
{\small 西湖大学} & {\small 2024.12} \\
{\small 麦吉尔大学 McGill University} & {\small 2024.11} \\
{\small 伍斯特理工学院 WPI} & {\small 2024.11} \\
{\small 德克萨斯大学奥斯汀分校 UT Austin} & {\small 2024.11} \\
{\small 德克萨斯农工大学 Texas A\&M University} & {\small 2024.11} \\
{\small 慕尼黑工业大学 TU Munich} & {\small 2024.11} \\
{\small 达姆施塔特工业大学《下一代机器人学习》研讨会} & {\small 2024.11} \\
{\small 斯坦福大学 Stanford University} & {\small 2024.10} \\
{\small 加州大学圣地亚哥分校 University of California, San Diego} & {\small 2024.10} \\
{\small 波士顿大学 Boston University} & {\small 2024.10} \\
{\small 宾夕法尼亚大学《GRASP SFI》研讨会 GRASP SFI Seminar, University of Pennsylvania} & {\small 2024.09} \\
{\small 华盛顿大学 University of Washington} & {\small 2024.09} \\
{\small 卡内基梅隆大学 Carnegie Mellon University} & {\small 2024.06} \\
{\small 布朗大学 Brown University} & {\small 2024.06; 2023.04} \\
{\small Boston Robotics演讲系列,由Universal Robots主办} & {\small 2023.03} \\
\end{tabular*}
\vspace{-3pt}

\resumeSubheading{《等变扩散策略》}{德国慕尼黑}
{机器人学习会议 (CoRL) 2024}{2024.11}
\resumeSubheading{《等变神经网络的前沿探索》(与Robin Walters合作)}{美国马萨诸塞州剑桥}
{MIT NeurReps全球演讲系列}{2024.10}
\resumeSubheading{《长期操作任务中的等变模型》}{美国马萨诸塞州剑桥}
{波士顿动力人工智能研究院}{2024.03}
\resumeSubheading{《潜在对称域中等变模型的惊人效果》}{卢旺达基加利}
{国际学习表征会议 (ICLR) 2023}{2023.05}
\resumeSubheading{《空间动作空间中的等变Q学习》}{美国纽约市}
{RSS 2022《机器人学习扩展》研讨会}{2022.06}
\resumeSubheading{《面向机器人操作的SO(2)等变强化学习》}{美国宾夕法尼亚州费城}
{ICRA 2022《机器人学习扩展》研讨会}{2022.05}
\resumeSubheading{《面向开放世界环境的辅助机器人抓取与放置》}{越南河内}
{国际机器人研究研讨会 (ISRR) 2019}{2019.12}
\vspace{1pt}
\resumeSubHeadingListEnd


%----------- SKILLS -----------

% \section{Skills}
% \resumeSubHeadingListStart
% \small{\item{
%     \textbf{Programming:} {Python, Java, C++} \\ \vspace{1pt}
    
%     \textbf{Tools:} {VSCode, PyCharm, IntelliJ IDEA, Git, LaTeX, Final Cut Pro } \\ \vspace{1pt}
    
%     \textbf{Robotics:} {Franka Emika, UR5, Baxter, Robotic Operating System (ROS), PyBullet, OpenRave}\\ \vspace{1pt}

%     \textbf{Machine Learning:} {PyTorch, NumPy}
    
%     % \textbf{Frameworks}{: X, X, X} \\
%     % \textbf{Developer Tools}{: X, X, X} \\
%     % \textbf{Libraries}{: X, X, X} \\
%     % \textbf{Applications}{: X, X, X}
% }}
% \resumeSubHeadingListEnd

%----------- CERTIFICATES -----------

% \section{Certificates}
  % \resumeSubHeadingListStart
    
    % \resumeOrganizationHeading
      % {Procter \& Gamble VIA Certificate Program}{Feb 2022}{Business Skills, Data and Digital Skills, Project Management and Personal Productivity}
    
  % \resumeSubHeadingListEnd



%----------- ORGANIZATIONS -----------

% \section{Organizations}
  % \resumeSubHeadingListStart
    
    % \resumeOrganizationHeading
      % {Institute of Electrical and Electronics Engineers (IEEE)}{Feb 2022 -- Present}{Student Member}
    
  % \resumeSubHeadingListEnd



%----------- HOBBIES -----------

% \section{Hobbies}
  % \resumeSubHeadingListStart
    % \small{\item{Swimming, Fitness, Eight-ball}}
  % \resumeSubHeadingListEnd



%----------- REFERENCES -----------

% \section{References}
  % \resumeSubHeadingListStart
    
  % \resumeSubHeadingListEnd



%-------------------------------------------

% \vfill
% \emph{Last updated {\today}.}

\end{document}
